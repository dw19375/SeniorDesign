\section{Introduction}
\subsection{Statement of Purpose}
Many homes and apartments suffer from large temperature differences from room to room. This can be due to many factors such as insulation, windows in the room, proximity to the kitchen, etc.  The problem is that the current standard thermostat only senses one room or area to determine the temperature of the whole house.  The Smart Climate Control System will allow anyone with a central heating and air conditioning (HVAC) system to have greater control over the temperatures in each room of the house.  Each room will have a remote temperature sensor and vent actuator, controlled by a main thermostat unit via wireless network.  The temperatures in each room of the house can then be dynamically controlled by the user interface much like standard thermostats.

\subsection{Goals}
\begin{itemize}
\item
Give user greater control over the climate in individual rooms in the building
\item
User can determine which room has priority so the temperature is more tightly controlled there.
\item
Non-priority rooms can have a wider temperature range so as to conserve energy.
\end{itemize}

\subsection{Functions}
\begin{itemize}
\item
Ability to set preferred temperature and priority room from main control panel
\item
Wireless communication with sensors in each room
\item
Wireless communication with vent actuators
\item
Sense temperature and humidity of each room
\end{itemize}

\subsection{Features}
\begin{itemize}
\item
Ability to prioritize rooms in the home as a reference for the thermostat
\item
Modular, scalable design
\item
Conserves energy by allowing a wider temperature range in non-priority rooms.
\item
Sets temperature based on heat index to account for humidity
\end{itemize}
