\section{Verification Procedures}
\emph{Please see Appendix~\ref{reqtab} for a complete listing of the design requirements.}
\subsection{Remote Temperature Sensor Unit}
\subsubsection{Power Supply}
To test the output voltage of the regulator we used a multimeter and tested the output both when the system was in sleep mode and when it was transmitting data. During both operation modes the voltage regulator had very little fluctuation from its output voltage value. The output voltage of the voltage regulator was measured to be $3.332 \pm 0.01\volts$ when no components were drawing power ($I=0\amps$) and under normal operating conditions, with $I\approx200\mamps$. 

\subsubsection{Sensor Data Collection}
\paragraph{Temperature Sensor}
The temperature was read from the temperature sensor and the temperature was reported via the LCD while debugging.  Since a common library was used to interface with the DS18B20 in all components, this only needed to be verified once.

\paragraph{Humidity Sensor}
The humidity sensor was not used in the final design.

\subsubsection{Sensor Accuracy}
\paragraph{Temperature Sensor}
% Fill in the thermometer specs
The room temperature was measured with a Fluke 80T-150U temperature probe and was compared to the reading of the temperature sensor.  The DS18B20 consistently read approximately $0.5^\circ$C higher than the Fluke temperature probe.  This was likely due to ambient heat from the temperature sensor unit.  This was not an issue on the main board, as the DS18B20 was better isolated from the electronics.

\paragraph{Humidity Sensor}
The humidity sensor was not used in the final design.

\subsubsection{Network}

\subsection{Vent Control Unit}
\subsubsection{Power Supply}
To test the output voltage of the regulator we used a multimeter and tested the output both when the system was in sleep mode and when the servo motor was in operation. During both times the voltage out of the regulator stayed fairly constant. The output voltage of the voltage regulator was measured to be $3.321 \pm 0.02\volts$ both when no components were drawing power ($I=0\amps$) and under normal operating conditions, with $I\approx200\mamps$. The $V_{\rm out-ripple}$ also passed its test.

The voltage from the battery pack was $6.32 \pm 0.1 \volts$, which satisfies the requirement that the input voltage $4.75 \leq V_{\rm in} \leq 26$. The maximum voltage for the servo motor is 6 volts, so a 0.7\volts\ diode was used to drop the voltage within the range of the servo motor. This final voltage to the servo motor was measured to be 5.65\volts .

\subsubsection{Servo Control}

\subsubsection{Network}

\subsection{Main Control Unit}

\subsubsection{Power Supply}
The updated power supply voltage outputs were measured with an oscilloscope.  Each had $<\pm .8VDC$ ripple.  The circuit was tested under load by powering the XBee, MSP430, and LCD under normal operation. No power or thermal problems noted after running for $>30$ minutes.

\subsubsection{Network}

\subsubsection{User Interface}
LCD turns on as required during main panel power up.  Brightness was not an issue with the $500\Omega$ resistor, and the screen was visible from $>4ft$ with default screen data as shown in the User Interface section.  The LCD displays the desired and current temperature, the priority room, and whether the system is heating or cooling as desired.

The main control unit properly receives packets from the PC application.  The main panel reacted to all the user desired entries.  The user can command any temperature in farenheit or celsius, and can set HEAT or COOL.  The main panel displayed these settings.

\subsubsection{HVAC Control Logic}
The HVAC system must respond to HEAT and COOL settings with respect to the current room temperatures.  The system was verified by using light bulbs to display when each system was commanded ON.  Using the PC app, we set the system to cool to $50^\circ F$, and the HVAC vents open with COOL signal sent to HVAC.  The system responded to a HEAT command by opening vents and sending the heat signal to HVAC when setting the system to $80^\circ F$.

Using a hair dryer in a small enclosed box, the heating stopped and the vents closed when the temperature reached $1^\circ F$ of set value.  In non-priority rooms, the vents close when the temperature is reached.  Cooling was checked similarly by using an icepack in the small box.

\subsubsection{HVAC Interface}
After changing the control circuitry, the heating and cooling are activated when appropriate by the main panel.  Light bulbs were used to verify that the main panel output the correct voltage at correct times to the Fan and Heat/Cool systems.


